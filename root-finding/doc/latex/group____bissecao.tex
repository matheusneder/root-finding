\hypertarget{group____bissecao}{
\subsection{Metodo da Bissecao}
\label{group____bissecao}\index{Metodo da Bissecao@{Metodo da Bissecao}}
}
\subsubsection*{Data Structures}
\begin{CompactItemize}
\item 
struct \hyperlink{structRootFindingBissecao}{RootFindingBissecao}
\begin{CompactList}\small\item\em Estrutura de dados para o Metodo da Bissecao. \item\end{CompactList}\end{CompactItemize}
\subsubsection*{Defines}
\begin{CompactItemize}
\item 
\#define \hyperlink{group____bissecao_g35fa37eaf728f4216a2b5b0ed5a9c685}{BISSECAO\_\-DEFAULT\_\-MAX\_\-ITERATIONS}~100
\item 
\#define \hyperlink{group____bissecao_g69c6773347f58386687f3b4bcdad0e01}{BISSECAO\_\-DEFAULT\_\-TOLERANCE}~1e-7
\end{CompactItemize}
\subsubsection*{Typedefs}
\begin{CompactItemize}
\item 
typedef struct \hyperlink{structRootFindingBissecao}{RootFindingBissecao} \hyperlink{group____bissecao_gb3511b238887380d8ad7579693f400d1}{RootFindingBissecaoT}
\begin{CompactList}\small\item\em Apelido para struct \hyperlink{structRootFindingBissecao}{RootFindingBissecao}. \item\end{CompactList}\end{CompactItemize}
\subsubsection*{Functions}
\begin{CompactItemize}
\item 
\hyperlink{structRootFindingBissecao}{RootFindingBissecaoT} $\ast$ \hyperlink{group____bissecao_g01fb79a4dd7e1f53eb1233f262528f66}{RootFindingBissecaoCreate} (\hyperlink{structRootFindingBase}{RootFindingBaseT} $\ast$rootsObj)
\begin{CompactList}\small\item\em Cria um objeto do tipo struct \hyperlink{structRootFindingBissecao}{RootFindingBissecao}. \item\end{CompactList}\item 
\hyperlink{RootFindingCommon_8h_31228d356f5429fa5ba7f206e4dee12f}{RootFindingBoolT} \hyperlink{group____bissecao_g565bfd11019354823afbcffe501801c8}{RootFindingBissecaoInit} (\hyperlink{structRootFindingBissecao}{RootFindingBissecaoT} $\ast$bissecaoObj)
\begin{CompactList}\small\item\em Inicializa o objeto \hyperlink{structRootFindingBissecao}{RootFindingBissecao}. \item\end{CompactList}\item 
void \hyperlink{group____bissecao_g9c2a72c616c6ae34254a9a807394ecb5}{RootFindingBissecaoDelete} (\hyperlink{structRootFindingBissecao}{RootFindingBissecaoT} $\ast$obj)
\begin{CompactList}\small\item\em Apaga a instancia do objeto \hyperlink{structRootFindingBissecao}{RootFindingBissecao}. \item\end{CompactList}\item 
\hyperlink{RootFindingCommon_8h_31228d356f5429fa5ba7f206e4dee12f}{RootFindingBoolT} \hyperlink{group____bissecao_g00f707bfd08d203eb0b941b6b09e5639}{RootFindingBissecaoPerformIteration} (\hyperlink{structRootFindingBissecao}{RootFindingBissecaoT} $\ast$bissecaoObj)
\begin{CompactList}\small\item\em Realiza a iteracao. \item\end{CompactList}\item 
void \hyperlink{group____bissecao_gee709dc3b98a74148de8312b12373bcc}{RootFindingBissecaoFindNewRange} (\hyperlink{structRootFindingBase}{RootFindingBaseT} $\ast$rootsObj)
\begin{CompactList}\small\item\em Encontra um novo intervalo \mbox{[}A, B\mbox{]} e os altera no objeto RootFindingBaseT baseado nos \mbox{[}A, B\mbox{]} e x existentes. Utilizado em \hyperlink{group____bissecao_g00f707bfd08d203eb0b941b6b09e5639}{RootFindingBissecaoPerformIteration} porem principalmente util para alterar o intervalo quando intercambiando entre metodos diferentes. \item\end{CompactList}\item 
int \hyperlink{group____bissecao_g9672d1ca4387db1792f8219968118900}{RootFindingBissecaoGetErrorCode} (\hyperlink{structRootFindingBissecao}{RootFindingBissecaoT} $\ast$bissecaoObj)
\begin{CompactList}\small\item\em Obtem o codigo de erro. \item\end{CompactList}\item 
int \hyperlink{group____bissecao_g2ab4fb7daf5901001d011ee85dc4cfe0}{RootFindingBissecaoGetStateCode} (\hyperlink{structRootFindingBissecao}{RootFindingBissecaoT} $\ast$bissecaoObj)
\begin{CompactList}\small\item\em Obtem o codigo referente ao estado do objeto. \item\end{CompactList}\item 
const char $\ast$ \hyperlink{group____bissecao_g77e94d3a9b5999461aabeca3bfe1837a}{RootFindingBissecaoGetErrorMessage} (\hyperlink{structRootFindingBissecao}{RootFindingBissecaoT} $\ast$bissecaoObj)
\begin{CompactList}\small\item\em Obtem a mensagem de erro. \item\end{CompactList}\item 
const char $\ast$ \hyperlink{group____bissecao_gb0455a1f4f30b2e8916d9dff5c237be1}{RootFindingBissecaoGetStateMessage} (\hyperlink{structRootFindingBissecao}{RootFindingBissecaoT} $\ast$bissecaoObj)
\begin{CompactList}\small\item\em Obtem a mensagem referente ao estado do objeto. \item\end{CompactList}\item 
\hyperlink{RootFindingCommon_8h_31228d356f5429fa5ba7f206e4dee12f}{RootFindingBoolT} \hyperlink{group____bissecao_gbcac5093ad9f3d46feb5d7eb89bb2a75}{RootFindingBissecaoHasError} (\hyperlink{structRootFindingBissecao}{RootFindingBissecaoT} $\ast$bissecaoObj)
\begin{CompactList}\small\item\em Verifica se ha erros. \item\end{CompactList}\item 
static void \hyperlink{group____bissecao_g35fd0bd3c36285504bfc64f6a4fc2727}{setError} (\hyperlink{structRootFindingBissecao}{RootFindingBissecaoT} $\ast$bissecaoObj, int errorCode)
\begin{CompactList}\small\item\em Set error code and change state to BISSECAO\_\-ERROR\_\-FOUND. \item\end{CompactList}\item 
static void \hyperlink{group____bissecao_g4dc0dd2ead3960f74572fc6d3afbe8a7}{resetError} (\hyperlink{structRootFindingBissecao}{RootFindingBissecaoT} $\ast$bissecaoObj)
\item 
static \hyperlink{RootFindingCommon_8h_a296fe63994e03408c4ad62794d472e9}{RootFindingDoubleT} \hyperlink{group____bissecao_g9edf187b4ec1c46191fc8d208f506247}{computeX} (\hyperlink{structRootFindingBissecao}{RootFindingBissecaoT} $\ast$bissecaoObj)
\begin{CompactList}\small\item\em Calcula o X baseado no intervalo \mbox{[}a, b\mbox{]}. \item\end{CompactList}\end{CompactItemize}


\subsubsection{Define Documentation}
\hypertarget{group____bissecao_g35fa37eaf728f4216a2b5b0ed5a9c685}{
\index{\_\-bissecao@{\_\-bissecao}!BISSECAO\_\-DEFAULT\_\-MAX\_\-ITERATIONS@{BISSECAO\_\-DEFAULT\_\-MAX\_\-ITERATIONS}}
\index{BISSECAO\_\-DEFAULT\_\-MAX\_\-ITERATIONS@{BISSECAO\_\-DEFAULT\_\-MAX\_\-ITERATIONS}!_bissecao@{\_\-bissecao}}
\paragraph[BISSECAO\_\-DEFAULT\_\-MAX\_\-ITERATIONS]{\setlength{\rightskip}{0pt plus 5cm}\#define BISSECAO\_\-DEFAULT\_\-MAX\_\-ITERATIONS~100}\hfill}
\label{group____bissecao_g35fa37eaf728f4216a2b5b0ed5a9c685}




Definition at line 34 of file RootFindingBissecao.h.

Referenced by RootFindingBissecaoCreate().\hypertarget{group____bissecao_g69c6773347f58386687f3b4bcdad0e01}{
\index{\_\-bissecao@{\_\-bissecao}!BISSECAO\_\-DEFAULT\_\-TOLERANCE@{BISSECAO\_\-DEFAULT\_\-TOLERANCE}}
\index{BISSECAO\_\-DEFAULT\_\-TOLERANCE@{BISSECAO\_\-DEFAULT\_\-TOLERANCE}!_bissecao@{\_\-bissecao}}
\paragraph[BISSECAO\_\-DEFAULT\_\-TOLERANCE]{\setlength{\rightskip}{0pt plus 5cm}\#define BISSECAO\_\-DEFAULT\_\-TOLERANCE~1e-7}\hfill}
\label{group____bissecao_g69c6773347f58386687f3b4bcdad0e01}




Definition at line 35 of file RootFindingBissecao.h.

Referenced by RootFindingBissecaoCreate().

\subsubsection{Typedef Documentation}
\hypertarget{group____bissecao_gb3511b238887380d8ad7579693f400d1}{
\index{\_\-bissecao@{\_\-bissecao}!RootFindingBissecaoT@{RootFindingBissecaoT}}
\index{RootFindingBissecaoT@{RootFindingBissecaoT}!_bissecao@{\_\-bissecao}}
\paragraph[RootFindingBissecaoT]{\setlength{\rightskip}{0pt plus 5cm}typedef struct {\bf RootFindingBissecao} {\bf RootFindingBissecaoT}}\hfill}
\label{group____bissecao_gb3511b238887380d8ad7579693f400d1}


Apelido para struct \hyperlink{structRootFindingBissecao}{RootFindingBissecao}. 



Definition at line 77 of file RootFindingBissecao.h.

\subsubsection{Function Documentation}
\hypertarget{group____bissecao_g9edf187b4ec1c46191fc8d208f506247}{
\index{\_\-bissecao@{\_\-bissecao}!computeX@{computeX}}
\index{computeX@{computeX}!_bissecao@{\_\-bissecao}}
\paragraph[computeX]{\setlength{\rightskip}{0pt plus 5cm}static {\bf RootFindingDoubleT} computeX ({\bf RootFindingBissecaoT} $\ast$ {\em bissecaoObj})\hspace{0.3cm}{\tt  \mbox{[}static, private\mbox{]}}}\hfill}
\label{group____bissecao_g9edf187b4ec1c46191fc8d208f506247}


Calcula o X baseado no intervalo \mbox{[}a, b\mbox{]}. 

\begin{Desc}
\item[Parameters:]
\begin{description}
\item[{\em bissecaoObj}]Ponteiro para o objeto \hyperlink{structRootFindingBissecao}{RootFindingBissecao} \end{description}
\end{Desc}
\begin{Desc}
\item[Returns:]O X calculado \end{Desc}


Definition at line 70 of file RootFindingBissecao.c.

References RootFindingBase::a, RootFindingBase::b, and RootFindingBissecao::rootsObj.

Referenced by RootFindingBissecaoInit(), and RootFindingBissecaoPerformIteration().\hypertarget{group____bissecao_g4dc0dd2ead3960f74572fc6d3afbe8a7}{
\index{\_\-bissecao@{\_\-bissecao}!resetError@{resetError}}
\index{resetError@{resetError}!_bissecao@{\_\-bissecao}}
\paragraph[resetError]{\setlength{\rightskip}{0pt plus 5cm}static void resetError ({\bf RootFindingBissecaoT} $\ast$ {\em bissecaoObj})\hspace{0.3cm}{\tt  \mbox{[}static, private\mbox{]}}}\hfill}
\label{group____bissecao_g4dc0dd2ead3960f74572fc6d3afbe8a7}




Definition at line 164 of file RootFindingBissecao.c.

References RootFindingBissecao::errorCode, and RootFindingBissecao::state.

Referenced by RootFindingBissecaoCreate(), RootFindingBissecaoInit(), RootFindingCordasCreate(), RootFindingCordasInit(), RootFindingNewtonRhapsonCreate(), RootFindingPegasoCreate(), and RootFindingPegasoInit().\hypertarget{group____bissecao_g01fb79a4dd7e1f53eb1233f262528f66}{
\index{\_\-bissecao@{\_\-bissecao}!RootFindingBissecaoCreate@{RootFindingBissecaoCreate}}
\index{RootFindingBissecaoCreate@{RootFindingBissecaoCreate}!_bissecao@{\_\-bissecao}}
\paragraph[RootFindingBissecaoCreate]{\setlength{\rightskip}{0pt plus 5cm}{\bf RootFindingBissecaoT}$\ast$ RootFindingBissecaoCreate ({\bf RootFindingBaseT} $\ast$ {\em rootsObj})}\hfill}
\label{group____bissecao_g01fb79a4dd7e1f53eb1233f262528f66}


Cria um objeto do tipo struct \hyperlink{structRootFindingBissecao}{RootFindingBissecao}. 

\begin{Desc}
\item[Parameters:]
\begin{description}
\item[{\em rootsObj}]Ponteiro para o objeto do tipo struct \hyperlink{structRootFindingBase}{RootFindingBase} \end{description}
\end{Desc}
\begin{Desc}
\item[Returns:]Ponteiro para o objeto criado \end{Desc}


Definition at line 47 of file RootFindingBissecao.c.

References BISSECAO\_\-DEFAULT\_\-MAX\_\-ITERATIONS, BISSECAO\_\-DEFAULT\_\-TOLERANCE, RootFindingBissecao::maxIterations, resetError(), RootFindingBissecao::rootsObj, RootFindingBissecao::state, and RootFindingBissecao::tolerance.\hypertarget{group____bissecao_g9c2a72c616c6ae34254a9a807394ecb5}{
\index{\_\-bissecao@{\_\-bissecao}!RootFindingBissecaoDelete@{RootFindingBissecaoDelete}}
\index{RootFindingBissecaoDelete@{RootFindingBissecaoDelete}!_bissecao@{\_\-bissecao}}
\paragraph[RootFindingBissecaoDelete]{\setlength{\rightskip}{0pt plus 5cm}void RootFindingBissecaoDelete ({\bf RootFindingBissecaoT} $\ast$ {\em obj})}\hfill}
\label{group____bissecao_g9c2a72c616c6ae34254a9a807394ecb5}


Apaga a instancia do objeto \hyperlink{structRootFindingBissecao}{RootFindingBissecao}. 

\begin{Desc}
\item[Parameters:]
\begin{description}
\item[{\em obj}]Ponteiro para o obj \hyperlink{structRootFindingBissecao}{RootFindingBissecao} \end{description}
\end{Desc}


Definition at line 90 of file RootFindingBissecao.c.\hypertarget{group____bissecao_gee709dc3b98a74148de8312b12373bcc}{
\index{\_\-bissecao@{\_\-bissecao}!RootFindingBissecaoFindNewRange@{RootFindingBissecaoFindNewRange}}
\index{RootFindingBissecaoFindNewRange@{RootFindingBissecaoFindNewRange}!_bissecao@{\_\-bissecao}}
\paragraph[RootFindingBissecaoFindNewRange]{\setlength{\rightskip}{0pt plus 5cm}void RootFindingBissecaoFindNewRange ({\bf RootFindingBaseT} $\ast$ {\em rootsObj})}\hfill}
\label{group____bissecao_gee709dc3b98a74148de8312b12373bcc}


Encontra um novo intervalo \mbox{[}A, B\mbox{]} e os altera no objeto RootFindingBaseT baseado nos \mbox{[}A, B\mbox{]} e x existentes. Utilizado em \hyperlink{group____bissecao_g00f707bfd08d203eb0b941b6b09e5639}{RootFindingBissecaoPerformIteration} porem principalmente util para alterar o intervalo quando intercambiando entre metodos diferentes. 

\begin{Desc}
\item[Parameters:]
\begin{description}
\item[{\em rootsObj}]Ponteiro para o objeto \hyperlink{structRootFindingBase}{RootFindingBase} \end{description}
\end{Desc}


Definition at line 139 of file RootFindingBissecao.c.

References RootFindingBase::a, RootFindingBase::b, RootFindingBase::fX, RootFindingBaseEval(), and RootFindingBase::x.

Referenced by RootFindingBissecaoPerformIteration().\hypertarget{group____bissecao_g9672d1ca4387db1792f8219968118900}{
\index{\_\-bissecao@{\_\-bissecao}!RootFindingBissecaoGetErrorCode@{RootFindingBissecaoGetErrorCode}}
\index{RootFindingBissecaoGetErrorCode@{RootFindingBissecaoGetErrorCode}!_bissecao@{\_\-bissecao}}
\paragraph[RootFindingBissecaoGetErrorCode]{\setlength{\rightskip}{0pt plus 5cm}int RootFindingBissecaoGetErrorCode ({\bf RootFindingBissecaoT} $\ast$ {\em bissecaoObj})}\hfill}
\label{group____bissecao_g9672d1ca4387db1792f8219968118900}


Obtem o codigo de erro. 



Definition at line 159 of file RootFindingBissecao.c.

References RootFindingBissecao::errorCode.\hypertarget{group____bissecao_g77e94d3a9b5999461aabeca3bfe1837a}{
\index{\_\-bissecao@{\_\-bissecao}!RootFindingBissecaoGetErrorMessage@{RootFindingBissecaoGetErrorMessage}}
\index{RootFindingBissecaoGetErrorMessage@{RootFindingBissecaoGetErrorMessage}!_bissecao@{\_\-bissecao}}
\paragraph[RootFindingBissecaoGetErrorMessage]{\setlength{\rightskip}{0pt plus 5cm}const char$\ast$ RootFindingBissecaoGetErrorMessage ({\bf RootFindingBissecaoT} $\ast$ {\em bissecaoObj})}\hfill}
\label{group____bissecao_g77e94d3a9b5999461aabeca3bfe1837a}


Obtem a mensagem de erro. 



Definition at line 180 of file RootFindingBissecao.c.

References RootFindingBissecao::errorCode, MSG\_\-BISSECAO\_\-NO\_\-ERROR, MSG\_\-BISSECAO\_\-X\_\-ISINF\_\-OR\_\-ISNAN\_\-ERROR, and MSG\_\-ROOTS\_\-UNKNOW\_\-ERROR.\hypertarget{group____bissecao_g2ab4fb7daf5901001d011ee85dc4cfe0}{
\index{\_\-bissecao@{\_\-bissecao}!RootFindingBissecaoGetStateCode@{RootFindingBissecaoGetStateCode}}
\index{RootFindingBissecaoGetStateCode@{RootFindingBissecaoGetStateCode}!_bissecao@{\_\-bissecao}}
\paragraph[RootFindingBissecaoGetStateCode]{\setlength{\rightskip}{0pt plus 5cm}int RootFindingBissecaoGetStateCode ({\bf RootFindingBissecaoT} $\ast$ {\em bissecaoObj})}\hfill}
\label{group____bissecao_g2ab4fb7daf5901001d011ee85dc4cfe0}


Obtem o codigo referente ao estado do objeto. 



Definition at line 173 of file RootFindingBissecao.c.

References RootFindingBissecao::state.\hypertarget{group____bissecao_gb0455a1f4f30b2e8916d9dff5c237be1}{
\index{\_\-bissecao@{\_\-bissecao}!RootFindingBissecaoGetStateMessage@{RootFindingBissecaoGetStateMessage}}
\index{RootFindingBissecaoGetStateMessage@{RootFindingBissecaoGetStateMessage}!_bissecao@{\_\-bissecao}}
\paragraph[RootFindingBissecaoGetStateMessage]{\setlength{\rightskip}{0pt plus 5cm}const char$\ast$ RootFindingBissecaoGetStateMessage ({\bf RootFindingBissecaoT} $\ast$ {\em bissecaoObj})}\hfill}
\label{group____bissecao_gb0455a1f4f30b2e8916d9dff5c237be1}


Obtem a mensagem referente ao estado do objeto. 



Definition at line 194 of file RootFindingBissecao.c.

References RootFindingBase::e, RootFindingBissecao::maxIterations, msg, MSG\_\-BISSECAO\_\-APPROXIMANTION\_\-ROOT\_\-FOUND, MSG\_\-BISSECAO\_\-ERROR\_\-FOUND, MSG\_\-BISSECAO\_\-INITIALIZED, MSG\_\-BISSECAO\_\-MAX\_\-ITERATIONS\_\-LIMIT\_\-REACHED, MSG\_\-BISSECAO\_\-NOT\_\-INIT, MSG\_\-ROOTS\_\-UNKNOW\_\-STATE, RootFindingBissecao::rootsObj, RootFindingBissecao::state, RootFindingBissecao::tolerance, and RootFindingBase::x.\hypertarget{group____bissecao_gbcac5093ad9f3d46feb5d7eb89bb2a75}{
\index{\_\-bissecao@{\_\-bissecao}!RootFindingBissecaoHasError@{RootFindingBissecaoHasError}}
\index{RootFindingBissecaoHasError@{RootFindingBissecaoHasError}!_bissecao@{\_\-bissecao}}
\paragraph[RootFindingBissecaoHasError]{\setlength{\rightskip}{0pt plus 5cm}{\bf RootFindingBoolT} RootFindingBissecaoHasError ({\bf RootFindingBissecaoT} $\ast$ {\em bissecaoObj})}\hfill}
\label{group____bissecao_gbcac5093ad9f3d46feb5d7eb89bb2a75}


Verifica se ha erros. 

\begin{Desc}
\item[Returns:]TRUE caso haja erro \end{Desc}


Definition at line 220 of file RootFindingBissecao.c.

References RootFindingBissecao::errorCode.\hypertarget{group____bissecao_g565bfd11019354823afbcffe501801c8}{
\index{\_\-bissecao@{\_\-bissecao}!RootFindingBissecaoInit@{RootFindingBissecaoInit}}
\index{RootFindingBissecaoInit@{RootFindingBissecaoInit}!_bissecao@{\_\-bissecao}}
\paragraph[RootFindingBissecaoInit]{\setlength{\rightskip}{0pt plus 5cm}{\bf RootFindingBoolT} RootFindingBissecaoInit ({\bf RootFindingBissecaoT} $\ast$ {\em bissecaoObj})}\hfill}
\label{group____bissecao_g565bfd11019354823afbcffe501801c8}


Inicializa o objeto \hyperlink{structRootFindingBissecao}{RootFindingBissecao}. 

\begin{Desc}
\item[Parameters:]
\begin{description}
\item[{\em bissecaoObj}]Ponteiro para o objeto a ser inicializado \end{description}
\end{Desc}
\begin{Desc}
\item[Returns:]TRUE em caso de sucesso \end{Desc}


Definition at line 77 of file RootFindingBissecao.c.

References computeX(), RootFindingBase::e, RootFindingBase::fX, RootFindingBissecao::i, infinity(), resetError(), RootFindingBaseEval(), RootFindingBissecao::rootsObj, RootFindingBissecao::state, TRUE, and RootFindingBase::x.\hypertarget{group____bissecao_g00f707bfd08d203eb0b941b6b09e5639}{
\index{\_\-bissecao@{\_\-bissecao}!RootFindingBissecaoPerformIteration@{RootFindingBissecaoPerformIteration}}
\index{RootFindingBissecaoPerformIteration@{RootFindingBissecaoPerformIteration}!_bissecao@{\_\-bissecao}}
\paragraph[RootFindingBissecaoPerformIteration]{\setlength{\rightskip}{0pt plus 5cm}{\bf RootFindingBoolT} RootFindingBissecaoPerformIteration ({\bf RootFindingBissecaoT} $\ast$ {\em bissecaoObj})}\hfill}
\label{group____bissecao_g00f707bfd08d203eb0b941b6b09e5639}


Realiza a iteracao. 

\begin{Desc}
\item[Parameters:]
\begin{description}
\item[{\em bissecaoObj}]Ponteiro para o objeto \end{description}
\end{Desc}
\begin{Desc}
\item[Returns:]TRUE caso haja mais iteracoes a serem realizadas \end{Desc}


Definition at line 95 of file RootFindingBissecao.c.

References computeX(), RootFindingBase::e, FALSE, RootFindingBase::fX, RootFindingBissecao::i, isInfOrNan(), RootFindingBissecao::maxIterations, RootFindingBaseEval(), RootFindingBissecaoFindNewRange(), RootFindingBissecao::rootsObj, setError(), RootFindingBissecao::state, RootFindingBase::state, RootFindingBissecao::tolerance, TRUE, and RootFindingBase::x.\hypertarget{group____bissecao_g35fd0bd3c36285504bfc64f6a4fc2727}{
\index{\_\-bissecao@{\_\-bissecao}!setError@{setError}}
\index{setError@{setError}!_bissecao@{\_\-bissecao}}
\paragraph[setError]{\setlength{\rightskip}{0pt plus 5cm}static void setError ({\bf RootFindingBissecaoT} $\ast$ {\em bissecaoObj}, \/  int {\em errorCode})\hspace{0.3cm}{\tt  \mbox{[}static, private\mbox{]}}}\hfill}
\label{group____bissecao_g35fd0bd3c36285504bfc64f6a4fc2727}


Set error code and change state to BISSECAO\_\-ERROR\_\-FOUND. 

\begin{Desc}
\item[Parameters:]
\begin{description}
\item[{\em bissecaoObj}]Ponteiro para objeto RootFindingBissecaoT \item[{\em errorCode}]Codigo de erro \end{description}
\end{Desc}


Definition at line 153 of file RootFindingBissecao.c.

References RootFindingBissecao::errorCode, and RootFindingBissecao::state.

Referenced by RootFindingBissecaoPerformIteration(), RootFindingCordasInit(), RootFindingCordasPerformIteration(), RootFindingNewtonRhapsonInit(), RootFindingNewtonRhapsonPerformIteration(), and RootFindingPegasoPerformIteration().